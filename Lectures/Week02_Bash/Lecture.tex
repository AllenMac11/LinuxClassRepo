\documentclass[10pt]{article}
\PassOptionsToPackage{hyphens}{url}
\usepackage{hyperref}
\usepackage[margin=0.75in]{geometry}

\usepackage{multicol}
\usepackage{textcomp}
\usepackage{color}
\usepackage{graphicx}
\definecolor{pblue}{rgb}{0.13,0.13,1}
\definecolor{pgreen}{rgb}{0,0.5,0}
\definecolor{pred}{rgb}{0.9,0,0}
\definecolor{pgrey}{rgb}{0.46,0.45,0.48}

\usepackage{listings}
\lstdefinestyle{term}{language=bash,
  columns=fullflexible,
  showspaces=false,
  showtabs=false,
  breaklines=true,
  showstringspaces=false,
  tabsize=2,
  breakatwhitespace=true,
  commentstyle=\color{pgreen},
  keywordstyle=\color{pblue},
  stringstyle=\color{pred},
  numbers=left,
  stepnumber=1,
  basicstyle=\small\ttfamily,
  frame=single,
  moredelim=[il][\textcolor{pgrey}]{$$},
  moredelim=[is][\textcolor{pgrey}]{\%\%}{\%\%},
  upquote=true
}
\lstdefinestyle{sh}{language=bash,
  columns=fullflexible,
  showspaces=false,
  showtabs=false,
  breaklines=true,
  showstringspaces=false,
  tabsize=2,
  breakatwhitespace=true,
  commentstyle=\color{pgreen},
  keywordstyle=\color{pblue},
  stringstyle=\color{pred},
  numbers=left,
  stepnumber=1,
  basicstyle=\small\ttfamily,
  frame=single,
  moredelim=[il][\textcolor{pgrey}]{$$},
  moredelim=[is][\textcolor{pgrey}]{\%\%}{\%\%},
  upquote=true
}

\lstdefinestyle{py}{language=python,
  columns=fullflexible,
  showspaces=false,
  showtabs=false,
  breaklines=true,
  showstringspaces=false,
  tabsize=2,
  breakatwhitespace=true,
  commentstyle=\color{pgreen},
  keywordstyle=\color{pblue},
  stringstyle=\color{pred},
  numbers=left,
  stepnumber=1,
  basicstyle=\small\ttfamily,
  frame=single,
  moredelim=[il][\textcolor{pgrey}]{$$},
  moredelim=[is][\textcolor{pgrey}]{\%\%}{\%\%},
  upquote=true
}

\lstdefinestyle{txt}{
  columns=fullflexible,
  showspaces=false,
  showtabs=false,
  breaklines=true,
  showstringspaces=false,
  tabsize=2,
  breakatwhitespace=true,
  numbers=left,
  stepnumber=1,
  basicstyle=\small\ttfamily,
  frame=single,
  moredelim=[il][\textcolor{pgrey}]{$$},
  moredelim=[is][\textcolor{pgrey}]{\%\%}{\%\%},
  upquote=true
}

\title{\textbf{Week 02} \\
More about the command line and shell scripting} 
\author{
	Melvyn Ian Drag
}
\date{\today}


\begin{document}
\maketitle

\begin{abstract}
For tonight's class we'll learn more about the Linux command line and shell
scripting.
\end{abstract}

\section*{Working on the Linux Command Line}
You know some commands already:
\begin{multicols}{4}
\begin{enumerate}
\item cat
\item ls
\item cd
\item cp
\item cut
\item cut -c
\item cut -b
\item echo
\item touch
\item \textbf{What else?}
\end{enumerate}
\end{multicols}

Here are some more commands that are useful:
\begin{multicols}{4}
\begin{enumerate}
\item sort
\item uniq
\item tee
\item history
\item grep 
\end{enumerate}
\end{multicols}

these commands are particularly useful in the context of \textbf{pipes}. Pipes
are one of the defining features of Unix systems, they are one of the things
that make unix great. A major tenet of Unix philosophy is that programs should
be very simple and do one thing - just one simple thing - but do it incredibly
efficiently and always get the correct result. Then these simple
programs can be strung together to make more interesting programs that are fast
and correct, because the tiny programs  (like cat and sort and ls and cd, etc. )
are fast and correct.

\section*{Pipes}
Consider the following file:

\lstinputlisting[style=txt]{Code/NumbersPipeExample/numbers.txt}

There are many things you can do with this file. By the way, I have it saved in
my current working directory as `numbers.txt' - you should do the same.

\begin{lstlisting}[style=term]
melvyn@laptop$ cat numbers.txt | sort
melvyn@laptop$ cat numbers.txt | sort -n
melvyn@laptop$ cat numbers.txt | sort -g
melvyn@laptop$ cat numbers.txt | sort | uniq
melvyn@laptop$ cat numbers.txt | uniq # note that this does not work.
\end{lstlisting}

So you see the idea of pipes? We used a few linux commands together, separated
by a `|' to achieve a cool result. I showed you above a few ways to sort, and
then the correct and the wrong way to find the uniq numbers in a file. 

Let's continue with this discussion of pipes:

\begin{lstlisting}[style=term]
melvyn@laptop$ echo hello | cut -b1-2
melvyn@laptop$ echo hello | cut -c1-4
melvyn@laptop$ echo "hello world" | cut -d" " -f1
melvyn@laptop$ te how cut with d and f is finicky
melvyn@laptop$ echo "hello      world" | cut -d" " -f2 # many spaces between hello and world.
melvyn@laptop$  default cut uses TAB ( \t ) as delimiter.
melvyn@laptop$ cat file  | md5sum
melvyn@laptop$ md5sum file
melvyn@laptop$ cat file | sha256
melvyn@laptop$ history | grep "that cool command I can only remember part of" # I do this at least 50 times a day
\end{lstlisting} 

...this bit about `cut' is super important.... let's play with it more to make
sure you understand it... heck... to make sure \textit{I} understand it.


\textbf{Spend 5 minutes playing with cut and pipes to extract various bits of
information}

\section*{stdin, stdout, stderr}

What a great time to be in a class. In the next few minutes you're going to hear
about what all the Linux people know. If you don't know what I'm about to say,
then you're a N00b and you'll probably ask annoying questions on the internet
that no one will want to answer.  



\subsection*{ 8:40 - 9:00 More vim commands to make your work better.}
\begin{itemize}
\item h, j, k, l
\item I prefer to just use the arrow keys
\item modes i, esc
\item quit with :q. To save and quit you use :wq or :x. If you want to know the difference, google the diff between :wq and :x. there is a slight difference, but it doesnt matter.
\item dd to delete a line
\item y is copy.
\item p is paste after the cursor
\item shift p is paste before the cursor
\item u is undo
\item CTRL + r is to redo.
\end{itemize}

Here is some more functionality that is so powerful and unique to vim that it
will make you feel scared. You've never had this power before and this is going
to make you think vim is hard. It's not. You can ignore the following for now,
but within 2 weeks when you've mastered the above you'll be ready to appreciate
how great the below commands are. I'm just showing you now to plan the seed so
that it can start to develop in your subconscious.

\begin{itemize}
\item w to go forward a word
\item e to go to the end of the next word
\item b to go to the previous word beginning.
\item to go forward 5 words, 5w
\item To go back 5 words, 5b.
\end{itemize}

\section*{ 9:00 - 9:05 Experiment with Vim}
Now is your time to try allthe things I've shown you and ask me questions if you
don't get it.

\section*{ 9:05 - 9:10 Bash Part 3 }

Return value of last command is : \$?. All these linux commands I've shown you
return information to the terminal. They tell the terminal if the command was
successful or not. To check the return code you type `echo \$?'.

\begin{lstlisting}[style=term]
melvyn@machine$ echo "hello world"
hello world 
melvyn@machine$? echo $?
0
melvyn@machine$ ehco "aksljdg"
# error
melvyn@machine$ echo $?
127
#the 127 indicates an error. There was an error because ehco is not a command,
echo is.
\end{lstlisting}

\section{ 9:10 - 9:30 More vim and the inevitable questions }
The coupe de gras. Registers in vim. Vim remembers the things you've yanked in the past. Type :reg
See that vim remembers the stuff you've put on your clipboard in the past.
Exercise
Type:
Hello
World
Foo
Bar
Baz
dd Hello
dd World
highlight and yank the other three
Then look at clip board.
Paste world
Paste Baz
Foo
Bar
Baz
Look at :reg and see what it remembers. I think it remembers the dd stuff but only remembers the latest yanked stuff.
Paste the things from your registers with <reg id>p e.g. ( when  in normal mode, not insert mode ) "1p, see what happens. 


\section*{ 9:30 - 9:45 Homework Discussion and further questions and answers }
Look at and discuss homework.
\end{document}
